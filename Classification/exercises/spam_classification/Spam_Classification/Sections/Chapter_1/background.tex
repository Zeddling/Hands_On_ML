\subsection{Background}
Spam emails are emails that are unsolicited and are sent in bulk to indiscriminate recipients (\cite{comodo_2021}). There are many different types of email spam but the most common are ads, chain letters, email spoofing, hoaxes, money scams, malware warnings and porn spam (\cite{gatefy_2021}). Email IDs are collected by automated software called spam bots which crawl the internet for them. Spam messages made up 47.3\% of the emails sent between 2014 and 2020 (\cite{johnson_2021}). Russia generated 23.52\%  of the global spam value making it the largest generator of spam emails (\cite{johnson_2021}). The spam email rate has been decreasing from over the years. However, a substantial portion of spam emails constitute emails of malicious intent constituting of spyware, trojans and ransomware (\cite{johnson_2021}).\\
Spam emails have been used to spread dangerous software. Examples of ransomware that were spread by spam emails were locky, troldesh, cryptolocker and petya among others (\cite{kaspersky_2021}). Some spam emails were used for phishing scams. 96\% of phishing scams are done by email (\cite{verizon}) thus showing the gravity of the situation.
\\
There are various methods used to detect spam emails some of which include looking for spelling mistakes, setting spam filters in front of the mail server and/or on the mail server and using spam detection software that use machine learning among other solutions (\cite{wikipedia_2021_email}).The common machine learning methods used to detect spam emails include KNeighbors, Random forests, Naive Bayes and Support Vector Machines with linear kernels among others. This paper seeks to investigate the various machine learning methods used to detect spam emails.