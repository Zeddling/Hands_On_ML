\subsubsection{Machine Learning Algorithms in Spam Email Detection}
Spam email filters are programs created to filter out spams from the user's inbox. There are various categories of filters that are used: content-based filters which include  word-based, heuristic and Bayesian filters which attempt to filter spam emails using the content of the email, list-based filters which attempt to filter spam using a predefined list of potential spam senders or legitimate senders, challenge/response system which force a sender to perform a task before sending the email to the recipient, collaborative filters that attempt to filter spam using a data collected from a pool of volunteers and DNS Lookup Systems that attempt to verify the domain name of the sender \cite{tewari_jangale_2016}. However some of these filters are prone to produce false positives or to not be aggressive as required.\\
Machine learning has been used to increase the efficiency of spam email filtering for example the Bayesian filter that learns what the user considers as spam and attempts to filter emails using the experience it has garnered. The two main learning styles used are supervised learning by the use of algorithms like KNN, Random Forest, Logistic regression etc and unsupervised learning by use of algorithms like the Apriori and k-Means.