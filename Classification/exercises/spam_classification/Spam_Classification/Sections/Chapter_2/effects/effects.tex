\subsection{Effects of Email Spam}
Spam email refers to any kind of unsolicited email that often comes in form of commercialized adverts (\cite{spam_jiss}). Email spam costs European companies \$2.8 billion in relation to lost productivity and US based companies \$20 billion as a result of purchasing more mail servers and storage equipment to be able to stabilize with the inflow of spam messages (\cite{spam_jiss}). In addition, more money is lost in time spent having staff free networks overloaded with spam. There are several dangers that are accompanied with email spam.\\
First and foremost is the prevalence of malware in the emails. Malware refers to any software designed to damage a computer, network or server (\cite{moir_2009}). It is a catch-all term that represents software like viruses, trojans etc. A study done under the Australian Internet Security Initiative in 2020 discovered that out of 21,131,389 unique emails obtained from participating Australian internet service providers, 0.54\% of the emails had attachments of which 31\% were identified to be compromised by malware. 10\% of the sample spam emails embedded with URLs directed to a compromised website (\cite{soam_malware}). Some of the common ransomwares identified by the study in the spam were Locky, NEMUCOD, Cerber, W97M and the Razy Trojan. Once infected with the ransomware, the infected is required to pay a certain amount before the file system can be decrypted. An example is an incident in 2016 when 3 hospitals were infected with the Locky ransomware (\cite{bbc_2016}). One of the hospitals was forced to pay \$17,000 for access to they're files.\\
According to the Cisco document :Email: Click with Caution", spam emails contribute to 96\% of the phishing scams (\cite{cisco_2019}). The most common types of attacks are: Office 365 phishing where the attacker tries to social engineer the victim into giving them they're Office 365 credentials, business emails compromise(BEC) where an attacker impersonates a C-level executive or above in order to trick the victim into performing a business function, digital extortion with the aim is to scare the victim into sending money to the attacker, packaging and invoice spam where the attacker impersonates a service and sends a false email to the victim with an invoice document infected with malware and advance fee fraud where the attacker sends an false email about a pending transfer which requires a certain amount to be sent before the transfer can be completed (\cite{cisco_2019}). In 2018, the document states that \$1.3 billion was lost to email phishing especially BEC and email account compromise methods.
\\
As a result of such loses, the need for better spam email detection was required in order to detect these emails before any damage could be caused.